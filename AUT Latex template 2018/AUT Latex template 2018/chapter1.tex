\chapter{مقدمه}
بر اساس قانون مور
\LTRfootnote{Moore's law}
قدرت پردازنده های کامپیوتر های کلاسیک هر دو سال، دو برابر میشود. اما این رویه تا حدی ادامه خواهد داشت که محدودیت های دنیای فیزیک کلاسیک به آن اجازه دهند. چرا که اندازه ی اعضای تشکیل دهنده ی پردازنده ها به حدی کوچک میشود که ناخودآگاه وارد فضای کوچک کوانتوم
\LTRfootnote{Quantum}
 میشوند. پیشبینی میشود این اتفاق در سال 2050 رخ دهد.
\\
 پیچیدگی محاسباتی
\LTRfootnote{Computational complexity}
 برخی الگوریتم ها در کامپیوتر های کلاسیک کمتر قابلیت کاهش ندارند. در حالی که کامپیوتر های کوانتومی، در تئوری میتوانند با مقدار بزرگی داده همانند یک واحد داده برخورد کنند و پیچیدگی محاسباتی الگوریتم ها را کاهش دهند.
\cite{singhbook1in2}
به طور کلی، محاسبات کوانتومی از کنش و واکنش مواد در جهان در سطح ذرات تشکیل دهنده ی آن بهره میگیرد و بر روی بستر پدیده ی نسبیت خاص
\LTRfootnote{Special relativity}
 پایه‌گذاری شده‌است. 
\\
برای مثال، کامپیوتر کلاسیک مشکلی در پیدا کردن نام فرد موردنظر در یک کتاب تلفن ندارند. اما برای مسائل ریاضی بهینه سازی پیچیده
\LTRfootnote{Complex mathematical optimizing}
  که مسائلی هستند که برای پیدا کردن حالت بهینه با توجه به متغیر های مختلف است، کامپیوتر های کلاسیک پاسخگو نیستند. از جمله این مسائل میتوان به اختصاص دادن منابع در ساخت یک برج بزرگ برای بدست آوردن کمترین خرج ممکن اشاره کرد.  چنین مسائلی در همه ی خوزه ها وجود دارند و کامپیوتر های کوانتومی برای اجرای این الگوریتم ها بسیار مناسب هستند. 
\cite{singhbook1in4}

‌\section{‌خواص دنیای محاسبات کوانتومی}\label{sec1}
\subsection{کیوبیت}
کیوبیت ها
\LTRfootnote{Qubits}
 در کامپیوتر های کوانتومی، معادل بیت ها
\LTRfootnote{bits}
 در کامپیوتر های کلاسیک هستند. یک بیت یا در حالت صفر قرار دارد یا در حالت یک قرار دارد. تفاوت کیوبیت ها در این است که میتوانند حالی به جز صفر یا یک داشته باشند یا میتوان گفت برهم‌نهی 
\LTRfootnote{superposition}
حالات را شاهد هستیم. درنتیجه، کیوبیت میتواند حالات بیشتری از بیت داشته باشد. هر کیوبیت، به یک احتمالی میتواند یک باشد و به یک احتمالی میتواند صفر باشد. 
\begin{equation}
\Psi = \alpha\left|0\right\rangle + \beta\left|1\right\rangle
\end{equation}
به طوری که 
$\alpha$
 و 
$\beta$
  شدت احتمال هستند و هر دو اعداد مختلط هستند به طوری که
\begin{equation}
\alpha^{2} + \beta^{2} = 1
\end{equation}
فضای حالتی که این دو متغیر تشکیل میدهند، یک فضای مختلط دو بعدی است.  حالات خاص صفر و یک، یک فضای بردار پایه ای
\LTRfootnote{orthonormal basis}
برای این فضای برداری تشکیل میدهند
\begin{equation}
\left|0\right\rangle = (0, 1) and \left|1\right\rangle = (1, 0)
\end{equation}
در شکل پایین، میتوانید کره ی بلاچ
\LTRfootnote{Bloch's sphere}
  که نوعی بازنمایی هندسی از حالت یک کیوبیت است، را مشاهده کنید.
\begin{figure}[!h]
\centerline{\includegraphics[width=.5\textwidth]{bloch.jpeg}}
\caption{بازنمایی کیوبیت در کره بلاچ}
\end{figure}

از آنجایی که یک پایان‌نامه یا رساله، یک نوشته بلند محسوب می‌شود، لذا اگر همه تنظیمات و مطالب پایان‌نامه را داخل یک فایل قرار بدهیم، باعث شلوغی
و سردرگمی می‌شود. به همین خاطر، قسمت‌های مختلف پایان‌نامه یا رساله  داخل فایل‌های جداگانه قرار گرفته است. مثلاً تنظیمات پایه‌ای کلاس، داخل فایل
\verb;AUTthesis.cls;، 
تنظیمات قابل تغییر توسط کاربر، داخل 
\verb;commands.tex;،
قسمت مشخصات فارسی پایان‌نامه، داخل 
\verb;fa_title.tex;,
مطالب فصل اول، داخل 
\verb;chapter1;
و ... قرار داده شده است. نکته مهمی که در اینجا وجود دارد این است که از بین این  فایل‌ها، فقط فایل 
\verb;AUTthesis.tex;
قابل اجرا است. یعنی بعد از تغییر فایل‌های دیگر، برای دیدن نتیجه تغییرات، باید این فایل را اجرا کرد. بقیه فایل‌ها به این فایل، کمک می‌کنند تا بتوانیم خروجی کار را ببینیم. اگر به فایل 
\verb;AUTthesis.tex;
دقت کنید، متوجه می‌شوید که قسمت‌های مختلف پایان‌نامه، توسط دستورهایی مانند 
\verb;input;
و
\verb;include;
به فایل اصلی، یعنی 
\verb;AUTthesis.tex;
معرفی شده‌اند. بنابراین، فایلی که همیشه با آن سروکار داریم، فایل 
\verb;AUTthesis.tex;
است.
در این فایل، فرض شده است که پایان‌نامه یا رساله شما، از5 فصل و یک پیوست، تشکیل شده است. با این حال، اگر
  پایان‌نامه یا رساله شما، بیشتر از 5 فصل و یک پیوست است، باید خودتان فصل‌های بیشتر را به این فایل، اضافه کنید. این کار، بسیار ساده است. فرض کنید بخواهید یک فصل دیگر هم به پایان‌نامه، اضافه کنید. برای این کار، کافی است یک فایل با نام 
\verb;chapter6;
و با پسوند 
\verb;.tex;
بسازید و آن را داخل پوشه 
\verb;AUTthesis;
قرار دهید و سپس این فایل را با دستور 
\texttt{\textbackslash include\{chapter6\}}
داخل فایل
\verb;AUTthesis.tex;
و بعد از دستور
\texttt{\textbackslash include\{chapter6\}}
 قرار دهید.

\section{نوع ناهنجاری}
قبل از هر چیز، بدیهی است که باید یک توزیع تِک مناسب مانند 
\verb;Live TeX;
و یک ویرایش‌گر تِک مانند
\verb;Texmaker;
را روی سیستم خود نصب کنید.  نسخه بهینه شده 
\verb;Texmaker;
را می‌توانید  از سایت 
 \href{http://www.parsilatex.com}{پارسی‌لاتک}%
\LTRfootnote{\url{http://www.parsilatex.com}}
 و
\verb;Live TeX;
را هم می‌توانید از 
 \href{http://www.tug.org/texlive}{سایت رسمی آن}%
\LTRfootnote{\url{http://www.tug.org/texlive}}
 دانلود کنید.
 
در مرحله بعد، سعی کنید که  یک پشتیبان از پوشه 
\verb;AUTthesis;
 بگیرید و آن را در یک جایی از هارددیسک سیستم خود ذخیره کنید تا در صورت خراب کردن فایل‌هایی که در حال حاضر، با آن‌ها کار می‌کنید، همه چیز را از 
 دست ندهید.
 
 حال اگر نوشتن پایان‌نامه اولین تجربه شما از کار با لاتک است، توصیه می‌شود که یک‌بار به طور سرسری، کتاب «%
\href{http://www.tug.ctan.org/tex-archive/info/lshort/persian/lshort.pdf}{مقدمه‌ای نه چندان کوتاه بر
\lr{\LaTeXe}}\LTRfootnote{\url{http://www.tug.ctan.org/tex-archive/info/lshort/persian/lshort.pdf}}»
   ترجمه دکتر مهدی امیدعلی، عضو هیات علمی دانشگاه شاهد را مطالعه کنید. این کتاب، کتاب بسیار کاملی است که خیلی از نیازهای شما در ارتباط با حروف‌چینی را برطرف می‌کند.
 
 
بعد از موارد گفته شده، فایل 
\verb;AUTthesis.tex;
و
\verb;fa_title;
را باز کنید و مشخصات پایان‌نامه خود مثل نام، نام خانوادگی، عنوان پایان‌نامه و ... را جایگزین مشخصات موجود در فایل
\verb;fa_title;
 کنید. دقت داشته باشید که نیازی نیست 
نگران چینش این مشخصات در فایل پی‌دی‌اف خروجی باشید. فایل 
\verb;AUTthesis.cls;
همه این کارها را به طور خودکار برای شما انجام می‌دهد. در ضمن، موقع تغییر دادن دستورهای داخل فایل
\verb;fa_title;
 کاملاً دقت کنید. این دستورها، خیلی حساس هستند و ممکن است با یک تغییر کوچک، موقع اجرا، خطا بگیرید. برای دیدن خروجی کار، فایل 
\verb;fa_title;
 را 
\verb;Save;، 
(نه 
\verb;As Save;)
کنید و بعد به فایل 
\verb;AUTthesis.tex;
برگشته و آن را اجرا کنید. حال اگر می‌خواهید مشخصات انگلیسی پایان‌نامه را هم عوض کنید، فایل 
\verb;en_title;
را باز کنید و مشخصات داخل آن را تغییر دهید.%
\RTLfootnote{
برای نوشتن پروژه کارشناسی، نیازی به وارد کردن مشخصات انگلیسی پروژه نیست. بنابراین، این مشخصات، به طور خودکار،
نادیده گرفته می‌شود.
}
 در اینجا هم برای دیدن خروجی، باید این فایل را 
\verb;Save;
کرده و بعد به فایل 
\verb;AUTthesis.tex;
برگشته و آن را اجرا کرد.

برای راحتی بیشتر، 
فایل 
\verb;AUTthesis.cls;
طوری طراحی شده است که کافی است فقط  یک‌بار مشخصات پایان‌نامه  را وارد کنید. هر جای دیگر که لازم به درج این مشخصات باشد، این مشخصات به طور خودکار درج می‌شود. با این حال، اگر مایل بودید، می‌توانید تنظیمات موجود را تغییر دهید. توجه داشته باشید که اگر کاربر مبتدی هستید و یا با ساختار فایل‌های  
\verb;cls;
 آشنایی ندارید، به هیچ وجه به این فایل، یعنی فایل 
\verb;AUTthesis.cls;
دست نزنید.

نکته دیگری که باید به آن توجه کنید این است که در فایل 
\verb;AUTthesis.cls;،
سه گزینه به نام‌های
\verb;bsc;,
\verb;msc;
و
\verb;phd;
برای تایپ پروژه، پایان‌نامه و رساله،
طراحی شده است. بنابراین اگر قصد تایپ پروژه کارشناسی، پایان‌نامه یا رساله را دارید، 
 در فایل 
\verb;AUTthesis.tex;
باید به ترتیب از گزینه‌های
\verb;bsc;،
\verb;msc;
و
\verb;phd;
استفاده کنید. با انتخاب هر کدام از این گزینه‌ها، تنظیمات مربوط به آنها به طور خودکار، اعمل می‌شود.

\section{ماهیت داده}
\subsection{انواع تصویربرداری}
همان‌طور که در بخش 
\ref{sec2}
گفته شد، برای جلوگیری از شلوغی و سردرگمی کاربر در هنگام حروف‌چینی، قسمت‌های مختلف پایان‌نامه از جمله فصل‌ها، در فایل‌های جداگانه‌ای قرار داده شده‌اند. 
بنابراین، اگر می‌خواهید مثلاً مطالب فصل ۱ را تایپ کنید، باید فایل‌های 
\verb;AUTthesis.tex;
و
\verb;chapter1;
را باز کنید و محتویات داخل فایل 
\verb;chapter1;
را پاک کرده و مطالب خود را تایپ کنید. توجه کنید که همان‌طور که قبلاً هم گفته شد، تنها فایل قابل اجرا، فایل 
\verb;AUTthesis.tex;
است. لذا برای دیدن حاصل (خروجی) فایل خود، باید فایل  
\verb;chapter1;
را 
\verb;Save;
کرده و سپس فایل 
\verb;AUTthesis.tex;
را اجرا کنید. یک نکته بدیهی که در اینجا وجود دارد، این است که لازم نیست که فصل‌های پایان‌نامه را به ترتیب تایپ کنید. می‌توانید ابتدا مطالب فصل ۳ را تایپ کنید و سپس مطالب فصل ۱ را تایپ کنید.

نکته بسیار مهمی که در اینجا باید گفته شود این است که سیستم
\lr{\TeX},
محتویات یک فایل تِک را به ترتیب پردازش می‌کند. به عنوان مثال، اگه فایلی، دارای ۴ خط دستور باشد، ابتدا خط ۱، بعد خط ۲، بعد خط ۳ و در آخر، خط ۴ پردازش می‌شود. بنابراین، اگر مثلاً مشغول تایپ مطالب فصل ۳ هستید، بهتر است
که دو دستور
\verb~\chapter{مقدمه}
بر اساس قانون مور
\LTRfootnote{Moore's law}
قدرت پردازنده های کامپیوتر های کلاسیک هر دو سال، دو برابر میشود. اما این رویه تا حدی ادامه خواهد داشت که محدودیت های دنیای فیزیک کلاسیک به آن اجازه دهند. چرا که اندازه ی اعضای تشکیل دهنده ی پردازنده ها به حدی کوچک میشود که ناخودآگاه وارد فضای کوچک کوانتوم
\LTRfootnote{Quantum}
 میشوند. پیشبینی میشود این اتفاق در سال 2050 رخ دهد.
\\
 پیچیدگی محاسباتی
\LTRfootnote{Computational complexity}
 برخی الگوریتم ها در کامپیوتر های کلاسیک کمتر قابلیت کاهش ندارند. در حالی که کامپیوتر های کوانتومی، در تئوری میتوانند با مقدار بزرگی داده همانند یک واحد داده برخورد کنند و پیچیدگی محاسباتی الگوریتم ها را کاهش دهند.
\cite{singhbook1in2}
به طور کلی، محاسبات کوانتومی از کنش و واکنش مواد در جهان در سطح ذرات تشکیل دهنده ی آن بهره میگیرد و بر روی بستر پدیده ی نسبیت خاص
\LTRfootnote{Special relativity}
 پایه‌گذاری شده‌است. 
\\
برای مثال، کامپیوتر کلاسیک مشکلی در پیدا کردن نام فرد موردنظر در یک کتاب تلفن ندارند. اما برای مسائل ریاضی بهینه سازی پیچیده
\LTRfootnote{Complex mathematical optimizing}
  که مسائلی هستند که برای پیدا کردن حالت بهینه با توجه به متغیر های مختلف است، کامپیوتر های کلاسیک پاسخگو نیستند. از جمله این مسائل میتوان به اختصاص دادن منابع در ساخت یک برج بزرگ برای بدست آوردن کمترین خرج ممکن اشاره کرد.  چنین مسائلی در همه ی خوزه ها وجود دارند و کامپیوتر های کوانتومی برای اجرای این الگوریتم ها بسیار مناسب هستند. 
\cite{singhbook1in4}

‌\section{‌خواص دنیای محاسبات کوانتومی}
\subsection{کیوبیت}
کیوبیت ها
\LTRfootnote{Qubits}
 در کامپیوتر های کوانتومی، معادل بیت ها
\LTRfootnote{bits}
 در کامپیوتر های کلاسیک هستند. یک بیت یا در حالت صفر قرار دارد یا در حالت یک قرار دارد. تفاوت کیوبیت ها در این است که میتوانند حالی به جز صفر یا یک داشته باشند یا میتوان گفت برهم‌نهی 
\LTRfootnote{superposition}
حالات را شاهد هستیم. درنتیجه، کیوبیت میتواند حالات بیشتری از بیت داشته باشد. هر کیوبیت، به یک احتمالی میتواند یک باشد و به یک احتمالی میتواند صفر باشد. 
\begin{equation}
\left|\Psi\right\rangle = \alpha\left|0\right\rangle + \beta\left|1\right\rangle = \begin{bmatrix}
 \alpha
\\
\beta
\end{bmatrix}
\end{equation}
به طوری که 
$\alpha$
 و 
$\beta$
  شدت احتمال هستند و هر دو اعداد مختلط هستند به طوری که
\begin{equation}
\alpha^{2} + \beta^{2} = 1
\end{equation}
فضای حالتی که این دو متغیر تشکیل میدهند، یک فضای مختلط دو بعدی است.  حالات خاص صفر و یک، یک فضای بردار پایه ای
\LTRfootnote{orthonormal basis}
برای این فضای برداری تشکیل میدهند.
\begin{equation}
\left|0\right\rangle = (0, 1) and \left|1\right\rangle = (1, 0)
\end{equation}
در شکل پایین، میتوانید کره ی بلاچ
\LTRfootnote{Bloch's sphere}
  که نوعی بازنمایی هندسی از حالت یک کیوبیت است، را مشاهده کنید.
\begin{figure}[!h]
\centerline{\includegraphics[width=.5\textwidth]{bloch.jpeg}}
\caption{بازنمایی کیوبیت در کره بلاچ}
\end{figure}
این بازنمایی را میتوانید به تعداد نامحدودی کیوبیت هم انطباق دهید. به طوری که با داشتن $n$ کیوبیت نیاز به نگهداری $n^{2}$ عدد خواهید داشت. این حالت زمانی رخ میدهد که $n$ کیوبیت درهم‌تنیده
\LTRfootnote{entangled}
 شوند به طوری که باهم یک حالت را تشکیل دهند و نتوان آن ها را جدا کرد. 
\cite{fundamentalsandapplications}
همچنان جمع مجذور همه ی مقادیر باید برابر با یک شود. نمایش انتزاعی دو کیوبیت به شکل زیر خواهد بود:
\begin{equation}
\left|\Psi\right\rangle = \alpha_{0}\left|00\right\rangle +  \alpha_{1}\left|01\right\rangle +  \alpha_{2}\left|10\right\rangle +  \alpha_{3}\left|11\right\rangle = \begin{bmatrix}
 \alpha_{0}
\\
 \alpha_{1}
\\
 \alpha_{2}
\\
 \alpha_{3}
\end{bmatrix}
\end{equation}
نمایش دو کیوبیت در فرم ماتریسی و دیراک
\LTRfootnote{Dirac}
:
\begin{equation}
\left|00\right\rangle  = \begin{bmatrix}
 \alpha_{1}
\\
 \alpha_{0}
\\
 \alpha_{0}
\\
 \alpha_{0}
\end{bmatrix}
\text{;}
\left|01\right\rangle  = \begin{bmatrix}
 \alpha_{0}
\\
 \alpha_{1}
\\
 \alpha_{0}
\\
 \alpha_{0}
\end{bmatrix}
\text{;}
\left|10\right\rangle  = \begin{bmatrix}
 \alpha_{0}
\\
 \alpha_{0}
\\
 \alpha_{1}
\\
 \alpha_{0}
\end{bmatrix}
\text{;}
\left|11\right\rangle  = \begin{bmatrix}
 \alpha_{0}
\\
 \alpha_{0}
\\
 \alpha_{0}
\\
 \alpha_{1}
\end{bmatrix}
\end{equation}


\subsection{ضرب تانسوری}
ضرب تانسوری
\LTRfootnote{Tensor product}
، عملیاتی است که بین دو ماتریس میتوان انجام داد. این عملیات، یکی از بخش های اصلی محاسبات کوانتومی است. برای اینکه بتوان سیستم های چند-کیوبیتی
\LTRfootnote{multiple-qubit systems}
را به صورت ریاضی نمایش داد، از این عملیات استفاده میشود. به این صورت که اگر $M$ یک ماتریس $(p,q)$ باشد و  $N$ یک ماتریس $(x,y)$ باشد، ماتریس ضرب تانسوری آنها یک ماتریس $(px,qy)$ خواهد بود. 
\cite{fundamentalsandapplications}
این ضرب را میتوان با یک گیت کوانتومی
\LTRfootnote{quantum gate}
 اعمال کرد.
\begin{equation}
M =  \begin{bmatrix}
 a_{11} &  a_{12}
\\
 a_{21} & a_{22}
\end{bmatrix}
\text{;}
N =  \begin{bmatrix}
 b_{11} &  b_{12}
\\
 b_{21} &  b_{22}
\end{bmatrix}
\end{equation}

\begin{equation}
M \oplus  N =  \begin{bmatrix}
 a_{11}b_{11} &  a_{11}b_{12} &  a_{12}b_{11} &  a_{12}b_{12}
\\
 a_{11}b_{21} &  a_{11}b_{22} &  a_{12}b_{21} &  a_{12}b_{22}
\\
 a_{21}b_{11} &  a_{21}b_{12} &  a_{22}b_{11} &  a_{22}b_{12}
\\
 a_{21}b_{21} &  a_{21}b_{22} &  a_{22}b_{21} &  a_{22}b_{22}
\end{bmatrix}
\end{equation}
برای ضرب تانسوری دو کیوبیت خواهیم داشت:
\begin{equation}
\left|0\right\rangle  \oplus  \left|1\right\rangle  =
\begin{bmatrix}
1 \\ 0 
\end{bmatrix} 
\oplus 
\begin{bmatrix}
0 \\ 1 
\end{bmatrix} 
=
  \begin{bmatrix}
0
\\
1
\\
0
\\
0
\end{bmatrix} = \left|01\right\rangle
\end{equation}
~
و
\verb~\chapter{الگوریتم‌های کوانتومی‌}
در حال حاضر، چندین الگوریتم کوانتومی وجود دارد. این الگوریتم‌ها، کیوبیت‌ها را به صورتی تغییر میدهند که مسائل را حل کنند. به طور کلی، این الگوریتم‌ها کارایی بالاتری از الگوریتم‌های کلاسیک معادل خود دارند و باید هم چنین باشد زیرا در غیر این صورت، استفاده از کامپیوترهای کلاسیک و الگوریتم‌های کلاسیک، گزینه بهتری خواهد بود. 
\\
همه الگوریتم‌های کوانتومی 4 مرحله ی یکسان را دنبال میکنند:
\begin{enumerate}
\item
هنگام شروع به کار سیستم، کیوبیت‌ها در یک حالت کلاسیک خاص قرار دارند(صفر یا یک)
\item
سیستم در حالت برهم‌نهی میرود
\item
سپس با اعمال گیت‌ها و عملیات مختلف، احتمالات برهم‌نهی تغییر پیدا میکند
\item
در نهایت، کیوبیت ها اندازه‌گیری میشوند
\cite{cambridgebook}
\end{enumerate}
در الگوریتم‌های کوانتومی، از آنجایی که کیوبیت‌ها و حالات دارای احتمال هستند، همیشه با ورودی‌های یکسان، خروجی یکسان نخواهد بود. با انجام یک الگوریتم بر روی یک ورودی مشخص بارها و بارها، میتوان احتمال هر خروجی را بدست آورد. معمولا، جواب درست، از لحاظ احتمالاتی، فاصله زیادی با دیگر خروجی‌ها دارد.
\section{الگوریتم دویچ}
الگوریتم دویچ
\LTRfootnote{Deutsch's algorithm}
ساده‌ترین الگوریتم کوانتومی است. فرض کنید تابعی داریم به نام $f$ که فضای ورودی آن $\{0, 1\}$ و فضای خروجی آن $\{0, 1\}$ است. در این تابع، اگر $f(0) != f(1)$ باشد، تابع متعادل است و اگر $f(0) = f(1)$ باشد، تابع ثابت است.
فرض کنید نمیدانیم تابع از کدام نوع است و میخواهیم نوع تابع را مشخص کنیم. در الگوریتم کلاسیک، برای بدست آوردن جواب، دو بار فراخوانی تابع نیاز است. اما با استفاده از الگوریتم دویچ و محاسبات کوانتومی، با یک بار فراخوانی تابع به جواب میرسیم. ابتدا مسئله را مدل میکنیم.
\ref{eq:deutsch}
\begin{equation} \label{eq:deutsch}
f(x)  \oplus f(y) = \left\{ \begin{array}{cl}
1 & : \ \verb; f is balanced; \\
0 & : \ \verb; f is constant;
\end{array} \right.
\end{equation}
تابع کوانتومی معادل با $f$ را میتوان با یک ماتریس نشان داد. این تابع برگشت‌پذیر است و دو کیوبیت دریافت میکند و حاصل آن نیز دو کیوبیت با احتمالا متفاوت است. در نمایش ماتریسی، ردیف بالا خروجی ها و ستون چپ، ورودی ها را نشان میدهد.
\ref{fig:deutschunitery}
\ref{fig:deutschmatrix}
\begin{figure}[!h]
\centerline{\includegraphics[width=.5\textwidth]{deutschunitery.jpeg}}
\caption{تابع کوانتومی ارزیابی $f$}
\label{fig:deutschunitery}
\end{figure}

\begin{figure}[!h]
\centerline{\includegraphics[width=.4\textwidth]{deutschmatrix.jpeg}}
\caption{بازنمایی ماتریسی تابع کوانتومی ارزیابی $f$}
\label{fig:deutschmatrix}
\end{figure}
حال، باید مداری طراحی کنیم که ابتدا دو کیوبیت جامد(در حالت ساده فیزیک) را به حالت برهم‌نهی ببرد، تابع ارزیابی را روی آنها اعمال کند و درنهایت، با تبدیل دوباره کیوبیت‌ها به حالت جامد، آنها را اندازه‌گیری کند. دویچ، مدار شکل
\ref{fig:deutschcircuit}
 را درنظر میگیرد.
\LTRfootnote{solid-state qubit}
\begin{figure}[!h]
\centerline{\includegraphics[width=.8\textwidth]{deutschcircuit.jpeg}}
\caption{مدار نهایی الگوریتم دویچ}
\label{fig:deutschcircuit}
\end{figure}
برای راحتی دنبال کردن چرایی کارایی این الگوریتم و آشنایی با معادلات کوانتومی، حالت کیوبیت‌ها در هر مرحله زمانی از الگوریتم به نمایش گذاشته‌شده است.
\ref{fig:deutschproof}

\begin{enumerate}\addtocounter{enumi}{-1}
\item
ورودی بالا را کیوبیت جامد با مقدار صفر و ورودی پایین را با مقدار یک قرار میدهیم.
\item
با اعمال گیت هادامار
\LTRfootnote{Hadamar's gate}
،
کیوبیت ها را به حالت برهم‌نهی میبریم. 
\item
در اینجا، مقدار های نامشخص $f$ را در معادله قرار میدهیم. درنهایت، جواب بدست آمده، این مقادیر را مشخص میکنند. در نتیجه، با استفاده از خروجی میتوانیم نوع تابع را مشخص کنیم.
\item
تابع هادامار برگشت‌پذیر است و با اعمال دوباره آن، تاثیر اولیه‌اش را از بین میبریم.
\end{enumerate}
\begin{figure}
\centerline{\includegraphics[width=1\textwidth]{deutschproof.jpg}}
\caption{حالت کیوبیت‌ها در هر مرحله از مدار}
\label{fig:deutschproof}
\end{figure}
و در نهایت، کیوبیت بالایی اندازه‌گیری میشود. اگر در حالت صفر قرار داشته باشد، تابع ثابت است و اگر در حالت یک باشد، تابع متعادل است. برتری الگوریتم کوانتومی این است که با یک بار ارزیابی، به جواب مسئله میرسیم که نصف قدم های الگوریتم کلاسیک است. هدف این الگوریتم این است که نشان دهد محاسبات کوانتومی میتوانند از محاسبات کلاسیک کارآمد تر باشند.
\cite{cambridgebook}

\section{الگوریتم گراور}
یکی از مسائل همیشگی کامپیوتر، پیدا کردن یک المان خاص در یک آرایه نامرتب با طول $m$ است. در حالت کلاسیک، برای اینکار در بدترین حالت، $m$ درخواست باید انجام دهیم و در حالت میانگین به $m/2$ درخواست نیاز است. الگوریتم گراور
\LTRfootnote{Grover's algorithm}
این زمان را به $\sqrt{m}$ درخواست تقلیل میدهد. البته چنین کاهش سرعتی در مقایسه با افزایش سرعت نمایی، چندان به چشم نمی‌آید. 
\\
جزئیات ریاضی این الگوریتم طولانی است درنتیجه در این بخش بیان نمیشود. اما کلیاتی از الگوریتم گراور را بیان میکنیم. ابتدا مسئله را مدل میکنیم. $x_{0}$ المان موردنظر است و آرایه هم طول $2^{n}$ دارد.
\ref{eq:grover}
\begin{equation} \label{eq:grover}
f(x) = \left\{ \begin{array}{cl}
1 & : if x = x_{0}  \\
0 & : if x \neq x_{0}
\end{array} \right.
\end{equation}
مراحل الگوریتم گراور برای آرایه با طول $2^{n}$:
\ref{fig:grovercircuit}
\begin{enumerate}
\item
تعداد $n$ کیوبیت با حالت صفر برمیداریم
\item
گیت هادامار $n$ تایی به رویشان اعمال میکنیم و کیوبیت‌ها را به حالت برهم‌نهی میبریم. 
\item
این مراحل را $\sqrt{2^{n}}$ بار تکرار میکنیم

\begin{enumerate}
\item
عملیات معکوس کردن فاز را انجام میدهیم: 
$U_{f}(I \otimes H)$
\item
عملیات معکوس میانگین را انجام میدهیم:
$ -I + 2A$
\end{enumerate}
\item
کیوبیت‌ها را اندازه‌گیری میکنیم.
\end{enumerate}
\begin{figure}
\centerline{\includegraphics[width=0.7\textwidth]{grovercircuit.jpeg}}
\caption{مدار گراور}
\label{fig:grovercircuit}
\end{figure}
\cite{cambridgebook}
\section{الگوریتم شور}
در سال 1994، پیتر شور
\LTRfootnote{Peter Shor}
با الهام از الگوریتم سایمون
\LTRfootnote{Simon's algorithm}
یک الگوریتم فاکتورگیری کوانتومی با پیچیدگی زمانی چندجمله ای خلق کرد. از سال 1970، محققان به دنبال الگوریتم‌های فاکتورگیری سریع‌تر هستند. پیچیدگی زمانی یک فاکتور مهم در سیستم‌های رمزنگاری است.
\cite{singhbook}
\\
بیشتر امنیت شبکه اینترنت بر مبنای پیچیدگی فاکتورگیری اعداد صحیح توسط کامپیوتر کلاسیک است. الگوریتم شور، به دلیل اهمیت و حساسیت زیاد باعث شد به حوزه محاسبات کوانتومی توجه بیشتری شود.
\cite{cambridgebook}
\\
بهترین الگوریتم فاکتورگیری دارای پیچیدگی زمانی
\begin{equation}
O(e^{cn^{1/3}\log^{2/3}n})
\end{equation}
($n = log_{2}N$ و $N$ عددی است که میخواهیم فاکتورگیری کنیم)
است. این درحالی است که پیچیدگی زمانی الگوریتم شور، 
\begin{equation}
O(n^{2}\log n \log \log n)
\end{equation}
است. که نسبت به $n$ چندجمله‌ای است.
\cite{singhbook}
\\
الگوریتم شور بر اساس یک حقیقت خلق شده است: مسئله فاکتورگیری را به مسئله یافتن تناوب یک تابع تبدیل کرد. 
\cite{cambridgebook}
شور برای پیدا کردن تناوب یک تابع از تبدیل فوریر کوانتومی
\LTRfootnote{quantum Frourier transformation}
بهره‌گیری میکند. همچنین، از توازی کوانتومی
\LTRfootnote{quantum parallelism}
برای ایجاد برهم‌نهی از تمام جواب‌های تناوب تابع استفاده میکند. در برخی از قدم‌های این الگوریتم، از محاسبات کلاسیک هم استفاده میشود.
\cite{singhbook}~
را در فایل 
\verb~AUTthesis.tex~،
غیرفعال%
\RTLfootnote{
برای غیرفعال کردن یک دستور، کافی است پشت آن، یک علامت
\%
 بگذارید.
}
 کنید. زیرا در غیر این صورت، ابتدا مطالب فصل ۱ و ۲ پردازش شده (که به درد ما نمی‌خورد؛ چون ما می‌خواهیم خروجی فصل ۳ را ببینیم) و سپس مطالب فصل ۳ پردازش می‌شود و این کار باعث طولانی شدن زمان اجرا می‌شود. زیرا هر چقدر حجم فایل اجرا شده، بیشتر باشد، زمان بیشتری هم برای اجرای آن، صرف می‌شود.

\subsection{انواع فرمت های تصویر}
برای وارد کردن مراجع به فصل 2
مراجعه کنید.
\subsection{آسیب های ممکن}
برای وارد کردن واژه‌نامه فارسی به انگلیسی و برعکس، بهتر است مانند روش بکار رفته در فایل‌های 
\verb;dicfa2en;
و
\verb;dicen2fa;
عمل کنید.
\nocite{*}
\section{انتخاب روش نهایی}
برای پرسیدن سوال‌های خود در مورد حروف‌چینی با زی‌پرشین،  می‌توانید به
 \href{http://forum.parsilatex.com}{تالار گفتگوی پارسی‌لاتک}%
\LTRfootnote{\url{http://www.forum.parsilatex.com}}
مراجعه کنید. شما هم می‌توانید روزی به سوال‌های دیگران در این تالار، جواب بدهید.
